\documentclass[UTF8]{ctexart}
\usepackage{lmodern}
\usepackage{amsmath}
\usepackage{amssymb}
\usepackage{graphicx}
\usepackage{geometry}
\usepackage{float}
\usepackage{color}
\geometry{left=2.18cm,right=2.18cm,top=1.54cm,bottom=2.0cm}
\pagestyle{empty}
\title{\textbf {2025春计算方法--实验报告 \#2}}
\author{\textbf{\color{red}仅供参考!}}
\date{\today}

\begin{document}

\maketitle

运行环境:[自己给出。。。]
%win11,vscode,py3

\section*{实验内容与要求}

    对函数\textcolor{blue}{$f(x)=\dfrac{3x+1}{x^2-2x+3},x\in[-5,5]$},构造其$N$次\textcolor{blue}{Lagrange插值函数} $p(x)$,取$\max\limits_{x\in[-5,5]}|f(x)-p(x)|\approx\max\limits_{0\leq i\leq 500}|f(y_i)-p(y_i)|,y_i=\dfrac{i}{50}-5$为\textcolor{blue}{近似误差}。其中\textcolor{red}{$N+1$}个\textcolor{red}{插值节点}取值为:

    \begin{enumerate}
        \item $x_i=-5+\dfrac{10}{N}i,i=0,1,\cdots,N$
        \item $x_i=-5\cos\left(\dfrac{2i+1}{2N+2}\pi\right),i=0,1,\cdots,N$(Chebyshev点)
    \end{enumerate}

    分别取\textcolor{blue}{$N=4,8,16,32$},比较以上两组节点的插值结果(保留到小数点后12位)


\section{数值结果}


    \begin{table}[H]
    \begin{center}
    \begin{tabular}{|c|c|c|}
    \hline
    $N$ & 方法1 & 方法2 \\
    \hline
    $4$ & $0.123456789012E+001$ & $0.123456789012E+001$ \\
    \hline
    $8$ &  &  \\
    \hline
    $16$ &  &  \\
    \hline
    $32$ &  &  \\
    \hline
    \end{tabular}
    \end{center}
    \end{table}


\section{算法分析}


\section{实验小结}

\end{document}