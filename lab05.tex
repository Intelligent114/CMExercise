\documentclass[UTF8]{ctexart}
\usepackage{lmodern}
\usepackage{amsmath}
\usepackage{amssymb}
\usepackage{graphicx}
\usepackage{xcolor}
\usepackage{geometry}
\usepackage{tikz}
\usepackage{caption}
\usepackage{float}
\usepackage{pgfplots}
\usepackage{mathtools}
\usepackage{multirow}
\usepackage{textcomp}


\usepackage{tabularx}
\geometry{left=2.18cm,right=2.18cm,top=1.94cm,bottom=2.0cm}
\pagestyle{empty}
\pgfplotsset{compat=1.18}
\title{\textbf {2025春计算方法--实验报告 \#5}}
\author{姓名:\underline{~~~~~~~~~~~}  学号:\underline{~~~~~~~~~~~~~~~~~} }
\date{\today}

\begin{document}

\maketitle

运行环境:\underline{~~~~~~~~~~~~}

\section*{实验内容与要求}

\textcolor{blue}{\textbf{利用最小二乘法预测未来人口}}

\textbf{下表为近若干年年末的人口数据}

\begin{table}[H]
    \centering
    \small
    \begin{tabular}{
      |>{\centering\arraybackslash}p{0.65cm}
      |>{\centering\arraybackslash}p{1.3cm}
      |>{\centering\arraybackslash}p{1.3cm}
      |>{\centering\arraybackslash}p{1.3cm}
      |>{\centering\arraybackslash}p{1cm}
      |>{\centering\arraybackslash}p{1cm}
      |>{\centering\arraybackslash}p{1.3cm}
      |>{\centering\arraybackslash}p{1.3cm}
      |>{\centering\arraybackslash}p{1.3cm}
      |}
    \hline
    \textbf{年份} & \textbf{年末人口(亿人)} & \textbf{出生人口(万人)} & \textbf{死亡人口(万人)} & \textbf{出生率(\textperthousand)} & \textbf{死亡率(\textperthousand)} & \textbf{城镇人口(亿人)} & \textbf{乡村人口(亿人)} & \textbf{城镇化率(\textperthousand)} \\
    \hline
    2017 & 14.0011 & 1723 & 986 & 12.31 & 7.04 & 8.4343 & 5.5688 & 60.2 \\ \hline
    2018 & 14.0541 & 1523 & 993 & 10.84 & 7.07 & 8.6433 & 5.4108 & 61.5 \\ \hline
    2019 & 14.1008 & 1465 & 998 & 10.39 & 7.05 & 8.8426 & 5.2582 & 62.7 \\ \hline
    2020 & 14.1212 & 1202 & 997.6 & 8.51 & 7.06 & 9.022 & 5.0992 & 63.9 \\ \hline
    2021 & 14.1260 & 1062 & 1014 & 7.52 & 7.18 & 9.1425 & 4.9835 & 64.7 \\ \hline
    2022 & 14.1175 & 956 & 1041 & 6.77 & 7.37 & 9.2071 & 4.9104 & 65.22 \\ \hline
    2023 & 14.1000 & 902 & 1110 & 6.40 & 7.87 & 9.3267 & 4.7733 & 66.15 \\ \hline
    2024 & 14.0541 & 954 & 1093 & 6.77 & 7.76 & 9.4350 & 4.6478 & 67.00 \\ \hline
    \end{tabular}
\end{table}

\textcolor{blue}{实验内容}:
利用\textcolor{blue}{最小二乘法}和\textcolor{blue}{次数不超过三次的多项式}函数,分别去拟合以上{\color{blue}总人口和出生人口数据};利用所构造出的拟合函数,分别去预测\textcolor{blue}{2025}, \textcolor{blue}{2030}, 以及\textcolor{blue}{2035}年年底的\textcolor{red}{\textbf{总人口}}(精确到小数点后3位,以\textcolor{red}{\textbf{亿}}为单位)和\textcolor{red}{\textbf{出生人口}}(精确到小数点后1位,以\textcolor{red}{\textbf{万}}为单位)。

\vspace{0.5cm}\textcolor{blue}{实验要求}:

\begin{enumerate}
    \item 分别\textcolor{blue}{画出\textbf{总人口和出生人口}的变化趋势图, 并列表给出人口预测的数值结果}。
    \item 结合我们的国情实际, 对当前的预测结果作出适当的点评, 并给出改进拟合的大致方案或思路。
\end{enumerate}

\newpage
\section{计算结果~~(请作图,并列表)}


\begin{itemize}
    \item Blah blah blah,

    \item Blah blah blah,

    \item Blah blah blah,
\end{itemize}

\section{算法分析}

    \begin{itemize}
        \item Blah blah blah,

        \item Blah blah blah,

        \item Blah blah blah,
    \end{itemize}

\section{实验小结}
{\color{blue}对本次实验, 作总结~(包括这次实验的体会, 收获, 预测结果点评, 出现的问题以及可能的改进方案或思路等等}.
\begin{itemize}
    \item Blah blah blah,
    \item Blah blah blah,
    \item Blah blah blah,

\end{itemize}


\end{document}




