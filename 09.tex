\documentclass[cn,hazy,green,11pt,normal]{elegantnote}
\title{计算方法作业\#9}

\author{陈文轩}
\institute{KFRC}

\date{\today}

\usepackage{amssymb}
\usepackage{float}
\usepackage{mathtools}

\newcommand*{\diff}{\mathop{}\!\mathrm{d}}



\begin{document}

\maketitle

\section{题目}


    \begin{enumerate}

    \item (4pts)设$\textcolor{blue}{n}$阶实方阵$\textcolor{blue}{A}$有相异的特征根$\textcolor{blue}{|\lambda_1|>|\lambda_2|>\cdots>|\lambda_n|>0}$。对给定的实数$\textcolor{blue}{\alpha\neq\lambda_i}$ ($\textcolor{blue}{i=1,2,\cdots,n}$),利用规范幂法或规范反幂法,设计一个能计算离$\textcolor{blue}{\alpha}$ \textcolor{red}{距离最近}的矩阵$\textcolor{blue}{A}$的特征根的迭代格式(注:不容许对矩阵求逆)。

    \item (8pts)考虑用Jacobi方法计算矩阵$\textcolor{blue}{A=\begin{bmatrix}7&1&2\\1&4&0\\2&0&3\end{bmatrix}}$的特征值。求对$\textcolor{blue}{A}$作一次Givens相似变换时的Givens(旋转)变换矩阵$\textcolor{blue}{Q}$(要求相应的计算效率最高)以及Givens变换后的矩阵$\textcolor{blue}{B}$(其中,$\textcolor{blue}{B=Q^{\top} AQ}$)。

    \item (8pts)设$\textcolor{blue}{p<q}$,$\textcolor{blue}{Q(p,q,\theta)}$为$\textcolor{blue}{n}$阶\textcolor{red}{Givens}矩阵,$\textcolor{blue}{\theta}$为角度。记\[\textcolor{blue}{A=(a_{ij})_{n\times n},B=(b_{ij})_{n\times n}=Q^{\top}(p,q,\theta)AQ(p,q,\theta)},\]假设$\textcolor{blue}{a_{pq}\neq0}$,证明:当$\textcolor{blue}{\theta}$满足$\textcolor{blue}{\cot2\theta= \dfrac{a_{qq}-a_{pp}}{2a_{pq}}}$时,有\[\textcolor{blue}{\sum_{i=1}^{n}b_{ii}^2=\sum_{i=1}^{n}a_{ii}^2+2a_{pq}^2.}\]\textbf{提示:}只需证$\textcolor{blue}{b_{pp}^2+b_{qq}^2=a_{pp}^2+a_{qq}^2+2a_{pq}^2}$。

    \item (10pts)设$\textcolor{blue}{A=\dfrac{1}{25}\begin{bmatrix}7&7&24\\0&50&-25\\24&24&-7\end{bmatrix}}$,利用Householder矩阵,求$\textcolor{blue}{A}$的正交分解,即$\textcolor{blue}{A=QR}$,其中$\textcolor{blue}{Q}$、$\textcolor{blue}{R}$分别为Householder正交阵和上三角阵。

\end{enumerate}

    Deadline:2025.5.5

\section{解答}

\end{document}