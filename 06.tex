\documentclass[cn,hazy,green,11pt,normal]{elegantnote}
\title{计算方法作业\#6}

\author{陈文轩}
\institute{KFRC}

\date{\today}

\usepackage{amssymb}
\usepackage{float}
\usepackage{mathtools}

\newcommand*{\diff}{\mathop{}\!\mathrm{d}}


\everymath{\displaystyle}

\begin{document}

\maketitle

\section{题目}

    \textbf{注意:\textcolor{red}{须给出解题过程或步骤,不可直接写答案;必要时,可使用计算器帮助。}}

    \textcolor{red}{$e=2.7182818285$}
    \begin{enumerate}
        \item (6pts) 利用牛顿迭代公式估算$\ln 2$的值(可取$f(x)=e^x-2=0$),取初值$x_0=0.618$,迭代$5$次,列表计算$x_i,i=1,2,\cdots,5$。请估计$x_5$的有效数字位数(计算$x_5$时,请保留尽量多的小数点位数);
        \item (6pts) 设$n>1$,给出用牛顿法计算$\sqrt[n]{a}(a>0)$时的迭代公式,并用它来计算$\sqrt[5]{2025}$,取初值$x_0=5.0$,求$x_4$;
        \item (10pts) 写出对方程$x^3-4x^2+5x-2=0$求根时的Newton迭代公式$x_n=\varphi(x_{n-1})$;
        \item (10pts) 设$f(x)$为$\mathbb{R}$上的光滑实值函数,$r\in\mathbb{R}$为$f(x)$的一个$p$重根($p\geq2$),试推导迭代公式$x_{k+1}=x_k-p\dfrac{f(x_k)}{f'(x_k)}$在根$r$附近的收敛阶。
    \end{enumerate}

    Deadline:2025.4.13

\section{解答}



\end{document}