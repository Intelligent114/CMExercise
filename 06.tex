\documentclass[cn,hazy,green,11pt,normal]{elegantnote}
\title{计算方法作业\#6}

\author{陈文轩}
\institute{KFRC}

\date{\today}

\usepackage{amssymb}
\usepackage{float}
\usepackage{mathtools}
\usepackage{extarrows}

\newcommand*{\diff}{\mathop{}\!\mathrm{d}}


\everymath{\displaystyle}

\begin{document}

\maketitle

\section{题目}

    \textbf{注意:\textcolor{red}{须给出解题过程或步骤,不可直接写答案;必要时,可使用计算器帮助。}}

    \textcolor{red}{$e=2.7182818285$}
    \begin{enumerate}
        \item (6pts) 利用牛顿迭代公式估算$\ln 2$的值(可取$f(x)=e^x-2=0$),取初值$x_0=0.618$,迭代$5$次,列表计算$x_i,i=1,2,\cdots,5$。请估计$x_5$的有效数字位数(计算$x_5$时,请保留尽量多的小数点位数);
        \item (6pts) 设$n>1$,给出用牛顿法计算$\sqrt[n]{a}(a>0)$时的迭代公式,并用它来计算$\sqrt[5]{2025}$,取初值$x_0=5.0$,求$x_4$;
        \item (10pts) 写出对方程$x^3-4x^2+5x-2=0$求根时的Newton迭代公式$x_n=\varphi(x_{n-1})$,取初值$x_0=0$,证明:$\lim\limits_{n\rightarrow\infty}x_n$存在;
        \item (10pts) 设$f(x)$为$\mathbb{R}$上的光滑实值函数,$r\in\mathbb{R}$为$f(x)$的一个$p$重根($p\geq2$),试推导迭代公式$x_{k+1}=x_k-p\dfrac{f(x_k)}{f'(x_k)}$在根$r$附近的收敛阶。
    \end{enumerate}

    Deadline:2025.4.13

\section{解答}

    $1.\,\,f(x)=e^x-2,f'(x)=e^x$,迭代公式为$x_{n+1}=x_n-\dfrac{e^{x_n}-2}{e^{x_n}}$。

    $\quad x_0=0.618,x_1=0.69600,x_2=0.693151,x_3=0.69314718056,x_4=0.6931471805599453,$

    $\quad x_5=0.69314718055994530941723212145818$。实际上,$x_5$的误差在$10^{-45}$量级。

    $2.\,\,f(x)=x^n-a,f'(x)=nx^{n-1}$,迭代公式为$x_{n+1}=x_n-\dfrac{x^n-a}{nx_n^{n-1}}$

    $\quad x_0=5,x_1=4.648,x_2=4.5858,x_3=4.58464,x_4=4.58443$。

    $3.\,\,f(x)=x^3-4x^2+5x-2,f'(x)=3x^2-8x+5$,迭代公式为$x_{n+1}=x_n-\dfrac{x_n^2-3x_n+2}{3x_n-5}$。

    $\quad \varphi(x)=\dfrac{2x^2-2x-2}{3x-5},\forall x\in[0,1),\dfrac{f(x)}{f'(x)}<0,2x^2-2x-2-(3x-5)>0\Rightarrow\varphi(x)<1$

    $\quad$因此从$x_0=0$开始的迭代序列是单调递增的,且有上界$1$,因此收敛。

    $4.\,\,$设$f(x)=(x-r)^p g(x),g(r\neq0),f'(x)=p(x-r)^{r-1}g(x)+(x-r)^p g'(x)$,令$e_k=x_k-r$,

    $\quad$则$r$附近$f(x_k)=e_k^{p}g(x_k),f'(x_k)=e_k^{p-1}(pg(x_k)+e_k g'(x_k))$,带入迭代公式,

    $\quad x_{k+1}=x_k-\dfrac{e_k^{p}g(x_k)}{e_k^{p-1}(pg(x_k)+e_k g'(x_k))}=x_k-\dfrac{e_k g(x_k)}{pg(x_k)+e_k g'(x_k)}$,两边同时减去$r$,\vspace{-0.5cm}

    \begin{flalign*}
        \qquad\quad e_{k+1}&=e_k-\dfrac{e_k g(x_k)}{pg(x_k)+e_k g'(x_k)}=e_k(1-\dfrac{pg(x_k)}{pg(x_k)+e_k g'(x_k)})=e_k\cdot\dfrac{e_k g'(x_k)}{pg(x_k)+e_k g'(x_k)}&\\
                            &=e_k^2\cdot\dfrac{g'(x_k)}{pg(x_k)+e_k g'(x_k)}\xlongequal{C\coloneqq \frac{g'(r)}{pg(r)}}Ce_k^2&
    \end{flalign*}

    $\quad$因此迭代公式是二次收敛的。
\end{document}