\documentclass[cn,hazy,green,11pt,normal]{elegantnote}
\title{计算方法作业\#2}

\author{陈文轩}
\institute{KFRC}

\date{\today}

\usepackage{amssymb}



\begin{document}

\maketitle


\section{题目}

    \textbf{注意:\textcolor{red}{须给出解题过程或步骤,不可直接写答案;必要时,可使用计算器帮助。}}

    \begin{enumerate}
        \item (6pts)利用下面的函数值表,作差商表,写出相应的牛顿插值多项式以及插值误差表达式,并计算$f(1.5)$和$f(4)$的近似值:
            \begin{table}[htb]
                \begin{center}
                    \begin{tabular}{|c|c|c|c|c|}
                    \hline
                    $x$ & $1.0$ & $2.0$ & $3.0$ & $4.5$ \\
                    \hline
                    $f(x)$ & $2.5$ & $4.0$ & $3.5$ & $2.0$ \\
                    \hline
                    \end{tabular}
                \end{center}
            \end{table}
        \item (6pts)利用数据$f(0)=2.0,f(1)=1.5,f(3)=0.25,f'(3)=1$构造出三次插值多项式,写出其插值余项,并计算$f(2)$的近似值。
        \item (6pts)设$f(x)=20x^3-x+2024$,求$f[1,2,4]$和$f[1,2,3,4]$;
        \item (6pts)设$\{l_i(x)\}_{i=0}^6$是以$\{x_i=2i\}_{i=0}^6$为节点的$6$次Lagrange插值基函数,求$\sum\limits_{i=0}^6 (x_i^3+x_i^2+1)l_i(x)$和$\sum\limits_{i=0}^6 (x_i^3+x_i^2+1)l_i'(x)$,结果需要化简。
        \item (6pts)设$x_0,x_1,\cdots,x_n(n>2)$为互异的节点,$l_k(x)(k=0,1,\cdots,n)$为与其对应的$n$次Lagrange插值基函数,证明$\sum\limits_{k=0}^n (x_k-x)^n l_k(x)=0$。
    \end{enumerate}

    截止日期:2025.3.16 23:59

    提交方式:通过bb系统提交

\section{解答}

    截止后更新。


\end{document}