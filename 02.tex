\documentclass[cn,hazy,green,11pt,normal]{elegantnote}
\title{计算方法作业\#2}

\author{陈文轩}
\institute{KFRC}

\date{\today}

\usepackage{amssymb}



\begin{document}

\maketitle


\section{题目}

    \textbf{注意:\textcolor{red}{须给出解题过程或步骤,不可直接写答案;必要时,可使用计算器帮助。}}

    \begin{enumerate}
        \item (6pts)利用下面的函数值表,作差商表,写出相应的牛顿插值多项式以及插值误差表达式,并计算$f(1.5)$和$f(4)$的近似值:
            \begin{table}[htb]
                \begin{center}
                    \begin{tabular}{|c|c|c|c|c|}
                    \hline
                    $x$ & $1.0$ & $2.0$ & $3.0$ & $4.5$ \\
                    \hline
                    $f(x)$ & $2.5$ & $4.0$ & $3.5$ & $2.0$ \\
                    \hline
                    \end{tabular}
                \end{center}
            \end{table}
        \item (6pts)利用数据$f(0)=2.0,f(1)=1.5,f(3)=0.25,f'(3)=1$构造出三次插值多项式,写出其插值余项,并计算$f(2)$的近似值。
        \item (6pts)设$f(x)=20x^3-x+2024$,求$f[1,2,4]$和$f[1,2,3,4]$;
        \item (6pts)设$\{l_i(x)\}_{i=0}^6$是以$\{x_i=2i\}_{i=0}^6$为节点的$6$次Lagrange插值基函数,求$\sum\limits_{i=0}^6 (x_i^3+x_i^2+1)l_i(x)$和$\sum\limits_{i=0}^6 (x_i^3+x_i^2+1)l_i'(x)$,结果需要化简。
        \item (6pts)设$x_0,x_1,\cdots,x_n(n>2)$为互异的节点,$l_k(x)(k=0,1,\cdots,n)$为与其对应的$n$次Lagrange插值基函数,证明$\sum\limits_{k=0}^n (x_k-x)^n l_k(x)=0$。
    \end{enumerate}

    截止日期:2025.3.16 23:59

    提交方式:通过bb系统提交

\section{解答}

    $1.\,\,$先计算各阶差商:$f[x_0]=2.5,f[x_1]=4,f[x_2]=3.5,f[x_3]=2;$

    $\quad\,\, f[x_0,x_1]=\frac{f[x_1]-f[x_0]}{x_1-x_0}=1.5,f[x_1,x_2]=\frac{f[x_2]-f[x_1]}{x_2-x_1}=-0.5,f[x_2,x_3]=\frac{f[x_3]-f[x_2]}{x_3-x_2}=-1;$

    $\quad\,\, f[x_0,x_1,x_2]=\dfrac{f[x_1,x_2]-f[x_0,x_1]}{x_2-x_0}=-1,f[x_1,x_2,x_3]=\dfrac{f[x_2,x_3]-f[x_1,x_2]}{x_3-x_1}=-0.2;$

    $\quad\,\, f[x_0,x_1,x_2,x_3]=\dfrac{f[x_0,x_1,x_2]-f[x_1,x_2,x_3]}{x_3-x_0}=\dfrac{8}{35}.$

    $\quad\,$因此,插值多项式$P_3(x)=2.5+1.5(x-1)-(x-1)(x-2)+\dfrac{8}{35}(x-1)(x-2)(x-3),$

    $\quad\,$完全展开可以得到$P_3(x)=\dfrac{8}{35}x^3-\dfrac{83}{35}x^2+\dfrac{491}{70}x-\dfrac{83}{35}.$

    $\quad\,$误差项$R_3(x)=\dfrac{f^{(4)}(\xi)}{24}(x-1)(x-2)(x-3)(x-4.5),\xi\in[1,4.5].$

    $\quad\,\, P_3(1.5)=\dfrac{251}{70},P_3(4)=\dfrac{83}{35}.$

    $2.\,\,$设插值多项式$P_3(x)=a_{3}x^3+a_2 x^2+a_1 x_1+a_0$,对应$P_3 '(x)=3a_3 x^2+2a_2 x+a_1;$

    $\quad\,$则有$P_3(0)=a_0=2,P_3(1)=a_3+a_2+a_1+a_0=1.5,P_3 '(3)=27a_3+6a_2+a_1=1$

    $\quad\,\, P_3(3)=27a_3+9a_2+3a_1+a_0=0.25\Rightarrow a_3=\dfrac{41}{144},a_2=-\dfrac{85}{72},a_1=\dfrac{19}{48},a_0=2$

    $\quad\,$所以插值函数为$P_3(x)=\dfrac{41}{144}x_3-\dfrac{85}{72}x^2+\dfrac{19}{48}x+2$,

    $\quad\,$余项为$R_3(x)=\dfrac{f^{(4)}(\xi)}{24}x(x-1)(x-3)^2,\xi\in[0,3];P_3(2)=\dfrac{25}{72}$

    $3.\,\,f[1]=2043,f[2]=2182,f[3]=2561,f[4]=3300;$

    $\quad\,\,f[1,2]=139,f[2,3]=379,f[3,4]=739,f[2,4]=559;$

    $\quad\,\,f[1,2,3]=120,f[2,3,4]=180,\color{red}{f[1,2,4]=140};$

    $\quad\,\,\color{red}{f[1,2,3,4]=20}.$

    $4.\,\,$记$f(x)=x^3+x^2+1$,则$l_i(x)$可以看作对$f(x)$插值时的基函数。由于节点数量为7,

    $\quad\,\,\deg f(x)=3$,所以$\sum\limits_{i=0}^6 (x_i^3+x_i^2+1)l_i(x)=f(x)=x^3+x^2+1,$

    $\quad\,\,\sum\limits_{i=0}^6 (x_i^3+x_i^2+1)l_i'(x)=f'(x)=3x^2+2x.$

    \begin{flalign*}
        \qquad 5.\,\, \sum_{k=0}^n (x_k-x)^n l_k(x) &= \sum_{k=0}^n \sum_{m=0}^n \left(\binom{n}{m}x_k^{n-m}(-x)^m\right) l_k(x)  & \\
                                                    &= \sum_{m=0}^n \binom{n}{m} (-x)^m \sum_{k=0}^n x_k^{n-m} l_k(x) =\sum_{m=0}^n \binom{n}{m}(-x)^m x^{n-m} & \\
                                                    &=(x-x)^n\equiv 0 & \\
    \end{flalign*}


\end{document}