\documentclass[cn,hazy,green,11pt,normal]{elegantnote}
\title{计算方法作业\#12}

\author{陈文轩}
\institute{KFRC}

\date{\today}

\usepackage{amssymb}
\usepackage{float}
\usepackage{mathtools}
\usepackage{textcomp}


\everymath{\displaystyle}

\newcommand*{\diff}{\mathop{}\!\mathrm{d}}


\begin{document}

\maketitle

\section{题目}


\begin{enumerate}

    \item (5pts)设$f(x)=x^2$,求$f(x)$在区间$[-\pi,\pi]$上的二次最佳平方逼近三角多项式。

    \item (10pts)设$f(x)\in C^2[a,b]$,且$f''(x)>0$。设$f(x)$在$[a,b]$上的一次最佳一致逼近多项式为$p_1^*(x)=c_0+c_1 x$。
        \begin{enumerate}
            \item 证明:$\exists c\in[a,b],\text{s.t.\,\,} c_1=f'(c)=\dfrac{f(b)-f(a)}{b-a},c_0=\dfrac{f(a)+f(c)}{2}-\dfrac{f(b)-f(a)}{b-a}\cdot\dfrac{a+c}{2}$;
            \item 求$f(x)=\cos x$在$\left[0,\dfrac{\pi}{2}\right]$上的一次最佳一致逼近多项式。
        \end{enumerate}

    \item (5pts)求多项式$p(x)=6x^3+3x^2+x+4$在$[-1,1]$上的二次最佳一致逼近多项式。

    \item (5pts)求函数$f(x)=\cos \dfrac{\pi}{2}x$在$[-1,1]$上关于权函数$\rho(x)=(1-x^2)^{-1/2}$的三次最佳平方逼近多项式。
\end{enumerate}

    Deadline:2025.5.25

\section{解答}

    $1.\,\,$即计算$f(x)$的Fourier级数,并截断到二次。由于$f(x)$是偶函数,故正弦系数为0。

    $\quad$余弦系数$a_n=\frac1{\pi}\int_{-\pi}^{\pi}x^2\cos nx \diff x=\dfrac{4(-1)^n}{n^2}$,常数项$a_0=\frac1{\pi}\int_{-\pi}^{\pi}x^2\diff x=\frac{\pi^2}{3}$,

    $\quad$故二次最佳平方逼近三角多项式为$S(x)=\frac{\pi^2}3-4\cos x+\cos 2x$。

    $2.\,\,$设$e(x)=f(x)-p_1^*(x)$,则$e''(x)=f''(x)>0$,故$e(x)$是凸函数。记$E=\min_{c_0,c_1}\max_{x\in[a,b]}|e(x)|$,

    $\quad$由Chebyshev,$e(a)=e(b)=-E,e(c)=E$,其中$c\in(a,b)$唯一存在。

    $\quad$此时有$\begin{cases}e(a)=f(a)-c_0-c_1 a=-E\\e(b)=f(b)-c_0-c_1 b=-E\end{cases}$,二者相减即得到$c_1=\dfrac{f(b)-f(a)}{b-a}$。

    $\quad$又有$e(c)=f(c)-c_0-c_1 c=E$,代入$e(a)=-E$即有$c_0=\dfrac{f(a)+f(c)}{2}-c_1\dfrac{a+c}{2}$,

    $\quad$且由于$e(x)$凸,故误差最大值点$c$满足$e'(c)=f'(c)-c_1=0$,即$c_1=f'(c)$。

    $\quad$由于$-\cos x$在$[0,\dfrac{pi}2]$凸,对$-\cos x$使用上述结论,得到其一次最佳一致逼近多项式为

    $\quad\tilde{p}_1^*(x)=\frac2{\pi}x-\dfrac{1+\sqrt{1-\dfrac{4}{\pi^2}}}2-\dfrac1{\pi}\arcsin \dfrac2{\pi}$,故$\cos x$的一次最佳一致逼近多项式为

    $\quad p_1^*(x)=-\frac2{\pi}x+\dfrac{1+\sqrt{1-\dfrac{4}{\pi^2}}}2+\dfrac1{\pi}\arcsin \dfrac2{\pi}\approx -0.6366x+1.1053$

    $3.\,\,$重写$f(x)=\dfrac32T_3(x)+3x^2+\dfrac{11}2x+4$,其中$T_3(x)=4x^3-3x$是3次Chebyshev多项式。

    $\quad\, T_3(x)$在$[-1,1]$上满足等振条件,故$f(x)$的二次最佳一致逼近多项式是$3x^2+\dfrac{11}2x+4$。

    $4.\,\,\,$Chebyshev多项式$\{T_i(x)\}$在权函数$\rho(x)$下正交,故所求$p(x)=\sum_{k=0}^3 \dfrac{\langle f(x),T_i(x) \rangle}{\langle T_i(x), T_i(x) \rangle}T_i(x)$。

    $\quad$由于$f(x)$和$\rho(x)$均为偶函数,故$T_1(x),T_3(x)$系数为0。考虑$T_0(x)=1,T_2(x)=2x^2-1$,

    $\quad$系数为$\alpha_0=\dfrac{\displaystyle\int_{-1}^1 \dfrac{\cos\tfrac{\pi}2 x}{\sqrt{1-x^2}}\diff x}{\displaystyle\int_{-1}^1 \dfrac1{\sqrt{1-x^2}}\diff x}=J_0\left(\dfrac{\pi}{2}\right)$和$\alpha_2=\dfrac{\displaystyle\int_{-1}^1 \dfrac{(2x^2-1)\cos\tfrac{\pi}2 x}{\sqrt{1-x^2}}\diff x}{\displaystyle\int_{-1}^1 \dfrac{(2x^2-1)^2}{\sqrt{1-x^2}}\diff x}=-2J_2\left(\dfrac{\pi}{2}\right)$。

    $\quad$故所求多项式为$p(x)=-4J_2\left(\dfrac{\pi}{2}\right)x^2+2J_2\left(\dfrac{\pi}{2}\right)+J_0\left(\dfrac{\pi}{2}\right)\approx -0.9988x^2+0.9714$。
\end{document}