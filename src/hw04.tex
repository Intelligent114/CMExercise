\chapter{第四次作业}

    \begin{homework}[6pts]
        给定函数$f(x)$离散值如下:
            \begin{table}[H]
                \begin{center}
                    \begin{tabular}{|c|c|c|c|c|}
                    \hline
                    $x$ & $0.00$ & $0.02$ & $0.04$ & $0.06$ \\
                    \hline
                    $f(x)$ & $2.5$ & $1.0$ & $2.0$ & $3.5$ \\
                    \hline
                    \end{tabular}
                \end{center}
            \end{table}
            分别用向前、向后以及中心差商公式计算$f'(0.02)$和$f'(0.04)$;
    \end{homework}

    \begin{solution}
        向前差分:$f'(0.02)=\dfrac{f(0.04)-f(0.02))}{0.02}=50,f'(0.04)=\dfrac{f(0.06)-f(0.04)}{0.02}=75;$

        向后差分:$f'(0.02)=\dfrac{f(0.02)-f(0.00)}{0.02}=-75,f'(0.04)=\dfrac{f(0.04)-f(0.02)}{0.02}=50;$

        中心差分:$f'(0.02)=\dfrac{f(0.04)-f(0.00)}{0.04}=-12.5,f'(0.04)=\dfrac{f(0.06)-f(0.02)}{0.04}=62.5$。
    \end{solution}

    \begin{homework}[8pts]
        用$3$点的Gauss-Legendre数值积分公式求积分$\int_0^2 e^{-x}\sin(x)\diff x$及其积分误差;
    \end{homework}

    \begin{solution}
        准确值:$0.4666$。先换元$\int_{-1}^1 e^{-t-1}\sin(t+1)\diff t$,记$g(t)=e^{-t-1}\sin(t+1)$,

        $I(g)=\dfrac59 g\left(-\sqrt{\dfrac35}\right)+\dfrac89 g(0)+\dfrac59 g\left(\sqrt{\dfrac35}\right)\approx0.4665$,
    \end{solution}

    \begin{homework}[8pts]
        试推导积分$\int_0^2 (x-1)^2 f(x)\diff x$的$2$点Gauss积分公式,这里$(x-1)^2$为权重函数;
    \end{homework}

    \begin{solution}
        先换元为$\int_{-1}^1 t^2 f(t+1)\diff t$,此时$\alpha_1=-\sqrt{\dfrac35},\alpha_2=\sqrt{\dfrac35}$,由于对称性,权重相等,代入

        $f(t)=1$时无误差,得到$W_1=W_2=\dfrac13$,故$I(f)=\dfrac13 f\left(1-\sqrt{\dfrac35}\right)+\dfrac13f\left(1+\sqrt{\dfrac35}\right)$。
    \end{solution}

    \begin{homework}[10pts]
        设函数$f(x)$充分光滑(可微),试推导如下数值微分公式(即确定常数$A,B,C,D,E$),使其截断误差为$O(h^4),f'(x)=\dfrac1h (Af(x-2h)+Bf(x-h)+Cf(x)+Df(x+h)+Ef(x+2h))$。
    \end{homework}

    \begin{solution}
    作Taylor展开至$O(h^5)$,有:

    $f(x+h)=f(x)+hf'(x)+\dfrac12 h^2 f''(x)+\dfrac16 h^3 f'''(x)+\dfrac1{24} h^4 f''''(x)+O(h^5)$

    $f(x-h)=f(x)-hf'(x)+\dfrac12 h^2 f''(x)-\dfrac16 h^3 f'''(x)+\dfrac1{24} h^4 f''''(x)+O(h^5)$

    $f(x+2h)=f(x)+2hf'(x)+2h^2 f''(x)+\dfrac43 h^3 f'''(x)+\dfrac23 h^4 f''''(x)+O(h^5)$

    $f(x-2h)=f(x)-2hf'(x)+2h^2 f''(x)-\dfrac43 h^3 f'''(x)+\dfrac23 h^4 f''''(x)+O(h^5)$\vspace{-0.9cm}

    \begin{flalign*}
        \qquad\qquad& Af(x-2h)+Bf(x-h)+Cf(x)+Df(x+h)+Ef(x+2h) &\\
        =&(A+B+C+D+E)f(x)+(-2A-B+D+2E)hf'(x)&\\
        &+(2A+\dfrac12 B+\frac12 D+2E)h^2 f''(x)+(-\dfrac 43 A-\dfrac16 B+\dfrac16 D+\dfrac43 E)h^3 f'''(x) &\\
        &+(\dfrac 23 A-\dfrac1{24} B+\dfrac1{24} D+\dfrac23 E)h^4 f''''(x)&\\
        \coloneqq& hf'(x)+O(h^5)&
    \end{flalign*}\vspace{-0.9cm}

    $\Rightarrow\begin{cases}A+B+C+D+E=0\\-2A-B+D+2E=1\\(2A+\dfrac12 B+\frac12 D+2E=0\vspace{0.2cm}\\-\dfrac 43 A-\dfrac16 B+\dfrac16 D+\dfrac43 E=0\vspace{0.2cm}\\\dfrac 23 A-\dfrac1{24} B+\dfrac1{24} D+\dfrac23E=0\end{cases}\Rightarrow\begin{cases}A=\dfrac1{12} \vspace{0.2cm}\\B=-\dfrac23\\ C=0\\D=\dfrac23\vspace{0.2cm}\\E=\dfrac1{12}\end{cases}$

    因此数值微分公式为$f'(x)=\frac{-f(x+2h)+8f(x+h)-8f(x-h)+f(x-2h)}{12h}+O(h^4)$。
    \end{solution}
