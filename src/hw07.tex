\chapter{第七次作业}

    \begin{homework}[10pts]
        用Doolittle分解法解如下线性方程组(请给出详细的解题过程,包括矩阵分解):
            \[\begin{cases}5x_1+x_2+2x_3&=2\\x_1+3x_2-x_3&=4\\2x_1+2x_2+5x_3&=10\end{cases}\]
    \end{homework}

    \begin{solution}
        对系数矩阵$A$作行变换:$r_2\leftarrow r_2-\frac15 r_1,r_3\leftarrow r_3-\frac25 r_1$,得到$U^{(1)}=\begin{bmatrix}5&1&2\\0&\tfrac{14}5&-\tfrac75\\0&\tfrac85&\tfrac{21}5\end{bmatrix},L^{(1)}=\begin{bmatrix}1&0&0\\\tfrac15&1&0\\\tfrac25&0&1\end{bmatrix}$,

        第二步行变换为$r_3\leftarrow r_3-\frac47 r_2$,得到$L^{(2)}=\begin{bmatrix}1&0&0\\\tfrac15&1&0\\\tfrac25&\tfrac47 &1\end{bmatrix}, U^{(2)}=\begin{bmatrix}5&1&2\\0&\tfrac{14}5&-\tfrac75\\0&0&5\end{bmatrix}$。

        因此系数矩阵的LU分解为$\begin{bmatrix}5&1&2\\1&3&-1\\2&2&5\end{bmatrix}=\begin{bmatrix}1&0&0\\\tfrac15&1&0\\\tfrac25&\tfrac47&1\end{bmatrix}\begin{bmatrix}5&1&2\\0&\tfrac{14}5&-\tfrac75\\0&0&5\end{bmatrix}$

        求解$Ly=b$,得到$y=\left(2,\dfrac{18}5,\dfrac{50}7\right)^{\top}$;求解$Lx=y$,得到$x=\left(-\dfrac47,2,\dfrac{10}7\right)^{\top}$。
    \end{solution}

    \begin{homework}[10pts]
        求如下三对角阵$A$的Crout分解:
            \[A=\begin{bmatrix}4&-1&0&0\\-1&4&-2&0\\0&-1&4&-2\\0&0&-1&4\end{bmatrix}\]
    \end{homework}

    \begin{solution}
        记$A=(a_{ij})_{4\times4}$,假设Crout分解是$\begin{bmatrix}4&-1&0&0\\-1&4&-2&0\\0&-1&4&-2\\0&0&-1&4\end{bmatrix}=\begin{bmatrix}l_{11}&0&0&0\\l_{21}&l_{22}&0&0\\l_{31}&l_{32}&l_{34}&0\\l_{41}&l_{42}&l_{43}&l_{44}\end{bmatrix}\begin{bmatrix}1&u_{12}&u_{13}&u_{14}\\0&1&u_{23}&u_{24}\\0&0&1&u_{34}\\0&0&0&1\end{bmatrix}$

        则$l_{11}=a_{11}=4,l_{21}=a_{21}=-1,l_{31}=a_{31}=0,l_{41}=a_{41}=0,u_{12}=\dfrac{a_{12}}{l_{11}}=-\frac14,u_{13}=\dfrac{a_{13}}{l_{11}}=0,$

        $u_{14}=\dfrac{a_{14}}{l_{11}}=0,l_{22}=a_{22}-l_{21}u_{12}=\dfrac{15}4,l_{32}=a_{32}-l_{31}u_{12}=-1,l_{42}=a_{42}-l_{41}u_{12}=0,$

        $u_{23}=\dfrac{a_{23}-l_{21}u_{13}}{l_{22}}=-\dfrac8{15},u_{24}=\dfrac{a_{24}-l_{21}u_{14}}{l_{22}}=0,l_{33}=a_{33}-(l_{31}u_{13}+l_{32}u_{23})=\dfrac{52}{15},$

        $l_{43}=a_{43}-(l_{41}u_{13}+l_{42}u_{23})=-1,u_{34}=\dfrac{a_{34}-(l_{31}u_{14}+l_{32}u_{24})}{l_{33}}=-\dfrac{15}{26}$

        $l_{44}=a_{44}-(l_{41}u_{14}+l_{42}u_{24}+l_{43}u_{34})=\dfrac{89}{26}$

        故Crout分解是$\begin{bmatrix}4&-1&0&0\\-1&4&-2&0\\0&-1&4&-2\\0&0&-1&4\end{bmatrix}=\begin{bmatrix}4&0&0&0\\-1&\frac{15}4&0&0\\0&-1&\frac{52}{15}&0\\0&0&-1&\frac{89}{26}\end{bmatrix}\begin{bmatrix}1&-\frac14&0&0\\0&1&-\frac8{15}&0\\0&0&1&-\frac{15}{26}\\0&0&0&1\end{bmatrix}$
    \end{solution}

    \begin{homework}[6+4pts]
        设有线性方程组
            \[\begin{cases}35.26x_1+14.96x_2&=20.25\\187.30x_1+79.43x_2&=19.75\end{cases}\]
        \begin{enumerate}
            \item 试求该方程组系数矩阵$A$的条件数$\mathrm{cond}_1(A)$(结果保留2位小数);
            \item 若方程组右端项$b=(20.25,19.75)^{\top}$有扰动$\delta b=(-0.01,0.01)^{\top}$,试给出此时方程组解的相对误差估计(在$\|\cdot\|_1$范数下,结果保留$2$位小数)。
        \end{enumerate}
    \end{homework}

    \begin{solution}
        $A=\dfrac1{100}\begin{bmatrix}3526&1496\\18730&7943\end{bmatrix},\|A\|_1=\dfrac{5564}{25},A^{-1}=\dfrac1{6531}\begin{bmatrix}-397150&74800\\936500&-176300\end{bmatrix}\approx\begin{bmatrix}-60.81&11.45\\143.39&-26.99\end{bmatrix},$

        $\|A^{-1}\|_1=\dfrac{444550}{2177}\approx204.20\Rightarrow \mathrm{cond}_1(A)=\|A\|_1\cdot\|A^{-1}\|_1=\dfrac{98939048}{2177}\approx 45447.43$。

        解的相对误差估计为$\dfrac{\|\delta x\|_1}{\|x\|_1}\lesssim\mathrm{cond}_1(A)\cdot\dfrac{\|\delta b\|_1}{\|b\|_1}\approx 22.7237$,即相对误差为$2272.37\%$。
    \end{solution}

