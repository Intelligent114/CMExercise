\chapter{第八次作业}

    \begin{homework}[10pts]
        设有线性方程组$Ax=b$,其中,\textcolor{blue}{\[A=\begin{bmatrix}2&-1&0&0\\-1&2&-1&0\\0&-1&2&-1\\0&0&-1& 2\end{bmatrix},\quad b=\begin{bmatrix}2\\0\\2\\4\end{bmatrix}\]}
        \begin{enumerate}
            \item 写出Jacobi迭代的迭代格式(\textcolor{red}{分量形式});
            \item 求Jacobi迭代的\textcolor{red}{迭代矩阵};
            \item 讨论此时Jacobi迭代法的收敛性(请给出理由或证明)。
        \end{enumerate}
    \end{homework}

    \begin{solution}
        迭代格式为$\begin{cases}x_1^{(k+1)}=\frac12(2+x_2^{(k)})\\x_2^{(k+1)}=\frac12(x_1^{(k)}+x_3^{(k)})\\x_3^{(k+1)}=\frac12(2+x_2^{(k)}+x_4^{(k)})\\x_4^{(k+1)}=\frac12(4+x_3^{(k)}\end{cases},G=D^{-1}(L+U)=\begin{bmatrix}0&\frac12&0&0\\\frac12&0 &\frac12&0 \\0&\frac12&0&\frac12\\0&0&\frac12& 0\end{bmatrix}$。

        $G$的特征值为$\dfrac{\pm 1\pm\sqrt{5}}{4},\rho(G)=\dfrac{1+\sqrt{5}}4<1$,故迭代收敛。
    \end{solution}

    \begin{homework}[10pts]
        设有线性方程组\textcolor{blue}{\[\begin{cases}5x_1-3x_2+2x_3&=10\\-3x_1+5x_2+2x_3&=20\\2x_1+2x_2+5x_3&=50\end{cases}\]}
        \begin{enumerate}
            \item 分别写出Gauss-Seidel迭代和SOR迭代的\textcolor{red}{分量形式};
            \item 求Gauss-Seidel迭代的分裂矩阵(splitting matrix)及迭代矩阵(iteration matrix);
            \item 讨论Gauss-Seidel迭代法的收敛性(请给出理由或证明)。
        \end{enumerate}
    \end{homework}

    \begin{solution}
        Gauss-Seidel迭代格式:$\begin{cases}x_1^{(k+1)}=\frac15(10+3x_2^{(k)}-2x_3^{(k)})\\x_2^{(k+1)}=\frac15(20+3x_1^{(k+1)}-2x_3^{(k)})\\x_1^{(k+1)}=\frac15(50-2x_1^{(k+1)}-2x_2^{(k+1)})\end{cases}$,

        SOR迭代格式:$\begin{cases}x_1^{(k+1)}=(1-\omega)x_1^{(k)}+\frac{\omega}5(10+3x_2^{(k)}-2x_3^{(k)})\\x_2^{(k+1)}=(1-\omega)x_2^{(k)}+\frac{\omega}5(20+3x_1^{(k+1)}-2x_3^{(k)})\\x_1^{(k+1)}=(1-\omega)x_3^{(k)}+\frac{\omega}5(50-x_1^{(k+1)}-2x_2^{(k+1)})\end{cases}$。

        分裂矩阵$Q=D+L=\begin{bmatrix}5&0&0\\-3&5&0\\2&2&5\end{bmatrix},G=-(D+L)^{-1}U=\dfrac1{125}\begin{bmatrix}0&75&-50\\0&45&-80\\ 0&-48&52\end{bmatrix}$。

        注意到系数矩阵各阶主子式为$\Delta_1=5,\Delta_2=\Delta_3=16$,是正定的,故迭代收敛。
    \end{solution}

    \begin{homework}[10pts]
        设有线性方程组$Ax=b$,其中,\textcolor{blue}{$A=\begin{bmatrix}3&2\\1&2\end{bmatrix},\quad b=\begin{bmatrix}3\\4\end{bmatrix}$}。

        利用如下迭代公式解此方程\textcolor{blue}{\[x^{(k+1)}=x^{(k)}+\alpha (b-Ax^{(k)}),\quad 0 \neq\alpha\in\mathbb{R}\]}
        \begin{enumerate}
            \item 写出此迭代法的迭代矩阵;
            \item 求使该迭代法收敛时参数$\alpha$的最大取值范围;
            \item 当$\alpha$取何值时,迭代收敛速度最快。
        \end{enumerate}
    \end{homework}

    \begin{solution}
        迭代公式可以写为$x^{(k+1)}=(I-\alpha A)x^{(k)}+\alpha b$,故迭代矩阵为$I-\alpha A=\begin{bmatrix}1-3\alpha&-2\alpha\\-\alpha &1-2\alpha\end{bmatrix}$。

        特征值为$1-\alpha,1-4\alpha$,收敛条件为$|1-\alpha|<1,|1-4\alpha|<1$,即$0<\alpha<\dfrac12$。

        当$\alpha=\dfrac25$时,谱半径$\rho(G)=\max\{|1-\alpha|,|1-4\alpha|\}$最小,收敛速度最快。
    \end{solution}

