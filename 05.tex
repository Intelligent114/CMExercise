\documentclass[cn,hazy,green,11pt,normal]{elegantnote}
\title{计算方法作业\#4}

\author{陈文轩}
\institute{KFRC}

\date{\today}

\usepackage{amssymb}
\usepackage{float}
\usepackage{mathtools}

\newcommand*{\diff}{\mathop{}\!\mathrm{d}}


\everymath{\displaystyle}



\begin{document}

\maketitle


\section{题目}

    \subsection{符号说明}

        对常微分方程$\dfrac{\diff y}{\diff x}=f(x,y)$,两边在区间$[x_{n-p},x_[n+1]]$上积分得$y(x_{n+1})=y(x_{n-p})+\int_{x_{n-p}}^{x_{n+1}}f(x,y)\diff x$。我们用数值积分来近似$\int_{x_{n-p}}^{x_{n+1}}f(x,y)\diff x$,从而构造线性多步格式。

        格式中有两个控制量$p$和$q$,其中$p$控制积分区间,$q$控制插值节点,若用积分节点$\{x_n,x_{n-1},$\\$\cdots,x_{n-q}\}$近似计算$\int_ {x_{n-p}}^{x_{n+1}}f(x,y)\diff x$,得到显式公式$y_{n+1}=y_{n-p}+\sum_{j=0}^q \beta_j f(x_{n-j},y_{n-j})$;若用积分节点$\{x_{n+1},x_n,\cdots,x_{n+1-q}\}$近似计算$\int_{x_{n-p}}^{x_{n+1}}f(x,y)\diff x$,得到隐式公式 $y_{n+1}=y_{n-p}+\sum_{j=-1}^{q-1} \beta_j f(x_{n-j},y_{n-j})$

        更一般地,一个$k+1$步的线性多步格式具有如下形式:
        \[y_{n+1}=\sum_{i=0}^k \alpha_i y_{n-i}+\sum_{j=-1}^k \beta_j f(x_{n-j},y_{n-j})\]

    \subsection{作业}

        \begin{enumerate}
            \item (12pts)设有常微分方程初值问题$\begin{cases}y'(x)=-y(x),0\leq x\leq 1\\y(0)=1\end{cases}$,假设求解区间$[0,1]$被$n$等分,令$h=\frac1n,x_k=\frac kn(k=0,1,\cdots,n)$
                \begin{enumerate}
                    \item 分别写出用\textcolor{red}{向前Euler公式,向后Euler公式,梯形公式以及改进的Euler公式}求上述微分方程数值解时的差分格式(即\textcolor{blue}{$y_{k+1}$与$y_k$}二者之间的递推关系式);
                    \item 设\textcolor{blue}{$y_0=y(0)$},分别求这四种公式(方法)下的近似值\textcolor{blue}{$y_n$}的表达式(注:这里的\textcolor{blue}{$y_n$}即是\textcolor{blue}{$y(x_n)\equiv y(1)$的近似值};
                    \item 当\textcolor{blue}{$n$}足够大(即区间长度\textcolor{blue}{$h\rightarrow 0$}时,分别判断四种方法下的近似值\textcolor{blue}{$y_n$}是否收敛到原问题的真解\textcolor{blue}{$y(x)$}在\textcolor{blue}{$x=1$}处的值。
                \end{enumerate}

            \item (8pts)试推导$p=1,q=2$显式公式\textcolor{blue}{$y_{n+1}=y_{n-1}+\dfrac h3\left(7f(x_n,y_n)-2f(x_{n-1},y_{n-1})+f(x_{n-2}\right.$}

                 \textcolor{blue}{$\left.,y_{n-2})\right)$}的\textcolor{red}{局部截断误差},即验证\textcolor{blue}{$T_{n+1}\equiv y(n+1)-y_{n+1}=\,\,$}\textcolor{magenta}{$\dfrac13 h^4 y^{(4)}(x_{n-1})+O(h^5)$}

                (提示:\textcolor{magenta}{将差分格式右端点某些项在某点处同时作Taylor展开});
            \item (18pts)试用线性多步法构造\textcolor{blue}{$p=1,q=2$}时的隐式差分格式,求该格式局部截断误差的\textcolor{red}{误差主项}并判断它的阶,最后为该隐式格式设计一种合适的预估-校正格式。
            \item (12pts)试推导如下Runge-Kutta公式的局部截断误差及其误差主项,判断该公式/格式的(精度)阶数。提示:\textcolor{magenta}{利用二元函数的Taylor展开。}
                \textcolor{blue}{\[\begin{cases}y_{n+1}=y_n+\dfrac h4(3k_1+k_2)\\k_1=f(x_n,y_n)\\k_2=f(x_n+2h,y_n+2hk_1)\end{cases}\]}

        \end{enumerate}

        Deadline:2025.4.6

\section{解答}



\end{document}