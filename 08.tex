\documentclass[cn,hazy,green,11pt,normal]{elegantnote}
\title{计算方法作业\#8}

\author{陈文轩}
\institute{KFRC}

\date{\today}

\usepackage{amssymb}
\usepackage{float}
\usepackage{mathtools}

\newcommand*{\diff}{\mathop{}\!\mathrm{d}}



\begin{document}

\maketitle

\section{题目}


    \begin{enumerate}
        
        \item (10pts)设有线性方程组$Ax=b$,其中,\textcolor{blue}{\[A=\begin{bmatrix}2&-1&0&0\\-1&2&-1&0\\0&-1&2&-1\\0&0&-1& 2\end{bmatrix},\quad b=\begin{bmatrix}2\\0\\2\\4\end{bmatrix}\]}
            \begin{enumerate}
                \item 写出Jacobi迭代的迭代格式(\textcolor{red}{分量形式});
                \item 求Jacobi迭代的\textcolor{red}{迭代矩阵};
                \item 讨论此时Jacobi迭代法的收敛性(请给出理由或证明)。
            \end{enumerate}

        \item (10pts)设有线性方程组\textcolor{blue}{\[\begin{cases}5x_1-3x_2+2x_3&=10\\-3x_1+5x_2+2x_3&=20\\2x_1+2x_2+5x_3&=50\end{cases}\]}
            \begin{enumerate}
                \item 分别写出Gauss-Seidel迭代和SOR迭代的\textcolor{red}{分量形式};
                \item 求Gauss-Seidel迭代的分裂矩阵(splitting matrix)及迭代矩阵(iteration matrix);
                \item 讨论Gauss-Seidel迭代法的收敛性(请给出理由或证明)。
            \end{enumerate}

        \item (10pts)设有线性方程组$Ax=b$,其中,\textcolor{blue}{$A=\begin{bmatrix}3&2\\1&2\end{bmatrix},\quad b=\begin{bmatrix}3\\4\end{bmatrix}$}。

            利用如下迭代公式解此方程\textcolor{blue}{\[x^{(k+1)}=x^{(k)}+\alpha (b-Ax^{(k)}),\quad 0 \neq\alpha\in\mathbb{R}\]}
            \begin{enumerate}
                \item 写出此迭代法的迭代矩阵;
                \item 求使该迭代法收敛时参数$\alpha$的最大取值范围;
                \item 当$\alpha$取何值时,迭代收敛速度最快。
            \end{enumerate}

    \end{enumerate}

    Deadline:2025.4.27

\section{解答}

\end{document}