\documentclass[cn,hazy,green,11pt,normal]{elegantnote}
\title{计算方法作业\#10}

\author{陈文轩}
\institute{KFRC}

\date{\today}

\usepackage{amssymb}
\usepackage{float}
\usepackage{mathtools}
\usepackage{textcomp}


\usepackage{tabularx}

\newcommand*{\diff}{\mathop{}\!\mathrm{d}}



\begin{document}

\maketitle

\section{题目}


    \begin{enumerate}

    \item (6pts)在最小二乘法原理下求下列矛盾方程组:$\begin{cases}x_1-2x_2&=4\vspace{-0.1cm}\\x_1+6x_2&=14\vspace{-0.1cm}\\3x_1+x_2&=7.5\vspace{-0.1cm}\\x_1+x_2&=4.5\end{cases}$

    \item (8pts)利用最小二乘法构造二次多项式 $y=p(x)$ 去拟合下列数据(这里$x$代表年份,$y$为人数),并计算 $y(2015)$,结果精确到小数点后一位。

        \begin{table}[H]
            \centering
            \begin{tabular}{|c|c|c|c|c|c|}
                \hline
                $x$ & 2010 & 2011 & 2012 & 2013 & 2014 \\
                \hline
                $y$ & 134091 & 134735 & 135404 & 136072 & 136782 \\
                \hline
            \end{tabular}
            \label{tab:1}
        \end{table}


    \item (6pts)给出下列数据,用最小二乘法求形如$y=a\cos x+b\sin x$的经验公式。

        \begin{table}[H]
            \centering
            \begin{tabular}{|c|c|c|c|c|}
                \hline
                $x_i$ & 0.20 & 0.25 & 0.30 & 0.50 \\
                \hline
                $y_i$ & 1.36 & 1.20 & 1.02 & 0.32 \\
                \hline
            \end{tabular}
            \label{tab:2}
        \end{table}


    \item (12pts)利用最小二乘法构造一个二次多项式$p(x)$,去拟合下列人口数据($x$代表年份,$p(x)$ 为人数,单位:亿),并分别预测一下\textcolor{magenta}{2024年末}和\textcolor{magenta}{2034年末}的人口数,计算结果精确到小数点后3位。

    \begin{table}[H]
        \centering
        \small
        \begin{tabular}{
          |>{\centering\arraybackslash}p{0.75cm}
          |>{\centering\arraybackslash}p{1.2cm}
          |>{\centering\arraybackslash}p{1.1cm}
          |>{\centering\arraybackslash}p{1.2cm}
          |>{\centering\arraybackslash}p{1cm}
          |>{\centering\arraybackslash}p{1cm}
          |>{\centering\arraybackslash}p{1.1cm}
          |>{\centering\arraybackslash}p{1.1cm}
          |>{\centering\arraybackslash}p{1.3cm}
          |}
        \hline
        \textbf{年份} & \textbf{年末人口} & \textbf{出生人口} & \textbf{死亡人口} & \textbf{出生率/\textperthousand} & \textbf{死亡率/\textperthousand} & \textbf{城镇人口} & \textbf{乡村人口} & \textbf{城镇化率/\textperthousand} \\
        \hline
        2018 & 14.0541 & 0.1523 & 0.0993 & 10.84 & 7.07 & 8.6433 & 5.4108 & 61.5 \\ \hline
        2019 & 14.1008 & 0.1465 & 0.0998 & 10.39 & 7.08 & 8.8426 & 5.2582 & 62.7 \\ \hline
        2020 & 14.1212 & 0.1202 & 0.09976 & 8.51 & 7.06 & 9.022 & 5.0992 & 63.9 \\ \hline
        2021 & 14.1260 & 0.1062 & 0.1014 & 7.52 & 7.18 & 9.1425 & 4.9835 & 64.7 \\ \hline
        2022 & 14.1175 & 0.0956 & 0.1041 & 6.77 & 7.37 & 9.2071 & 4.9104 & 65.22 \\ \hline
        2023 & 14.0967 & 0.0902 & 0.1110 & 6.40 & 7.87 & 9.3267 & 4.7733 & 66.15 \\ \hline
        \end{tabular}
        \label{tab:3}
    \end{table}
\end{enumerate}

    Deadline:2025.5.11

\section{解答}

\end{document}